\chapter{Conclusions}

In this study a wind turbine was studied using a BEM theory code to produce results for tip speed ratios of 6, 8, and 10, over yaw angles 0, 15, and 30 degrees. The turbine has three blades, a radius of $50$ meters, and a twist and blade pitch specified in section 1.1. The blades had a constant airfoil, the DU 95-W-180, for which lift and drag polars were provided. The BEM code includes Prandtl's tip correction and Glauerts correction for heavily loaded blades and yawed rotors. For the yawed cases the data is averaged over the azimuth so that it is easier to compare to the non-yawed cases.

The results from chapter 3 show that the results get worse for increasing yaw angles. As the yaw angle gets larger the average angle of attack over the rotor goes down, the axial and azimuthal inductions decrease, and the axial and azimthal forces decrease. These changes hold for all tip speed ratios. The behavior and shape of the plots of the non-yawed and yawed results are the same however, meaning that yawing the rotor does not introduce any new or strange phenomena that significantly affect the results. The yawing does mean that the result are now not constant over the azimuth as shown in section 3.2. The yaw angle also affects the wake of the rotor, however, in this study the wake was not analyzed. To analyze the wake the vortex cylinder model can be used. 

For optimal use of the wind turbine the different modes of operation must be compared. Since any yaw angle negatively affects the thrust, torque, and power the wind turbine should always aim to point directly into the flow when possible. To compare the different tip speed ratios the resulting thrust, torque and power can be compared. For increasing tip speed ratio the thrust increases while the torque decreases. However, since the power is the product of the torque and rotational speed the power increases for higher tip speed ratios. This means the most power is generated at a tip speed ratio of 10 and no yaw.

%SHORT discussion/conclusion, including the similarities and differences between the two rotor configurations (yaw vs. aligned rotor ), flow field and operation