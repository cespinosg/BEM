\chapter{Results}


Describe the initial conditions, table, cool


\section{BEM alligned rotor \textcolor{red}{BERNAT}}


\subsection{Main outputs \textcolor{red}{BERNAT}}


\subsubsection{Angle of attack and inflow angle \textcolor{red}{BERNAT}}

\subsubsection{Axial and azimuthal inductions \textcolor{red}{BERNAT}}

\subsubsection{Thrust and azimuthal loading \textcolor{red}{BERNAT}}

\subsubsection{Total thrust and torque \textcolor{red}{BERNAT}}


\section{BEM yawed rotor \textcolor{blue}{NIKLAS}}

The results in this section were obtained for the rotor operating with tip speeds $\lambda = 6,8,10$ and yaw values of $\psi = 15, 30$ degrees. 

\subsection{Main outputs \textcolor{blue}{NIKLAS}}


\subsubsection{Angle of attack and inflow angle \textcolor{blue}{NIKLAS}}
\begin{figure}[htbp]
	\centering
	\includegraphics[height=0.45\textheight]{./img/yaw/alpha_phi-yaw_15.pdf}
	\caption{Angle of attack and inflow angle at a yaw of 15 degrees.}
	\label{img:yaw-aoa-15}
\end{figure}
\begin{figure}[htbp]
	\centering
	\includegraphics[height=0.45\textheight]{./img/yaw/alpha_phi-yaw_30.pdf}
	\caption{Angle of attack and inflow angle at a yaw of 30 degrees.}
	\label{img:yaw-aoa-30}
\end{figure}

\subsubsection{Axial and azimuthal inductions \textcolor{blue}{NIKLAS}}
\begin{figure}[htbp]
	\centering
	\includegraphics[height=0.45\textheight]{./img/yaw/a_aip-yaw_15.pdf}
	\caption{Axial and azimuthal inductions at yaw of 15 degrees.}
	\label{img:yaw-aap-15}
\end{figure}
\begin{figure}[htbp]
	\centering
	\includegraphics[height=0.45\textheight]{./img/yaw/a_aip-yaw_30.pdf}
	\caption{Axial and azimuthal inductions at yaw of 30 degrees.}
	\label{img:yaw-aap-30}
\end{figure}

\subsubsection{Thrust and azimuthal loading \textcolor{blue}{NIKLAS}}
\begin{figure}[htbp]
	\centering
	\includegraphics[height=0.45\textheight]{./img/yaw/Fax_Faz-yaw_15.pdf}
	\caption{Thrust and azimuthal loading normalized by $\frac{1}{2} \rho U_\infty^2 R$.}
	\label{img:yaw-f-15}
\end{figure}
\begin{figure}[htbp]
	\centering
	\includegraphics[height=0.45\textheight]{./img/yaw/Fax_Faz-yaw_30.pdf}
	\caption{Thrust and azimuthal loading normalized by $\frac{1}{2} \rho U_\infty^2 R$.}
	\label{img:yaw-f-30}
\end{figure}

\subsubsection{Total thrust and torque \textcolor{blue}{NIKLAS}}
\begin{table}[]
\caption{CT and CP for yaw angle of 15 degrees.}
\begin{tabular}{|l|l|l|l|}
\hline
   & tsr 6 & tsr 8 & tsr 10  \\ \hline
CT &   0.4789    &  0.6359     &   0.7467     \\ \hline
CP &   0.3438    &  0.4234     &   0.4367     \\ \hline
\end{tabular}
\end{table}
\begin{table}[]
\caption{CT and CP for yaw angle of 30 degrees.}
\begin{tabular}{|l|l|l|l|}
\hline
   & tsr 6 & tsr 8 & tsr 10  \\ \hline
CT &   0.4404    &  0.5751     &   0.6796     \\ \hline
CP &   0.2837    &  0.3443     &   0.3569     \\ \hline
\end{tabular}
\end{table}

\section{Influence of the tip correction \textcolor{green}{CARLOS}}

The results shown in this section were obtained for the rotor described in the assignment instructions, operating with a tip speed ratio $ \lambda = 8 $ and no yaw. It can be seen that the tip correction reduces the power and thrust, resulting in a worse performance. Indeed, the rotor without the tip correction has a higher $ C_P/C_T $ ratio.

Near the blade tip, the flow angle $ \phi $ is reduced due to the tip vortex (Figure \ref{img:tc-phi}), because it induces a larger axial velocity (Figure \ref{img:tc-a}). Having a lower flow angle $ \phi $ results in a reduced power extraction, which is proportional to $ c_l \sin \phi - c_d \cos \phi $ (Figure \ref{img:tc-dcp-dmu}) \cite{weh-ch3}. However, reducing the flow angle $ \phi $ contributes to an increase of the thrust, since it is proportional to $ c_l \cos \phi + c_d \sin \phi $.

Note that $ c_l $ also decreases due to the reduction of $ \phi $, because it implies a decrease of the angle of attack $ \alpha $ (Figure \ref{img:tc-alpha}). Moreover, the relative velocity, which also affects the loads, will also be larger for the case without tip correction.

The tip correction tries to account for this effect. The expression Prandtl derived for that factor is shown in equation \ref{eq:f-prandtl}. Its value over the blade is plotted in Figure \ref{img:tc-f}. It relates the induction factor near the blade $ a_b $ with the azimuth average $ a $: $ a_b = a/f $ \cite{weh-ch3}.

\begin{equation}
f(\mu) = \frac{2}{\pi} \arccos \left[ \exp \left( - \frac{B}{2} \left( \frac{1-\mu}{\mu} \right) \sqrt{1+\frac{\lambda^2\mu^2}{(1-a)^2}} \right) \right]
\label{eq:f-prandtl}
\end{equation}

\begin{itemize}
	
	\item Power coefficient $ C_P $
	\begin{itemize}
		\item Tip correction: 0.4528
		\item No tip correction: 0.4757
		\item Increase: 5.05 \%
	\end{itemize}
	
	\item Thrust coefficient $ C_T $
	\begin{itemize}
		\item Tip correction: 0.6581
		\item No tip correction: 0.6691
		\item Increase: 1.67 \%
	\end{itemize}
	
	\item Power to thrust ratio $ C_P/C_T $
	\begin{itemize}
		\item Tip correction: 0.6880
		\item No tip correction: 0.7109
		\item Increase: 3.32 \%
	\end{itemize}
	
\end{itemize}

\begin{figure}[htbp]
	\centering
	\includegraphics[height=0.45\textheight]{./img/tip-correction/phi.pdf}
	\caption{Flow angle distribution with and without tip correction.}
	\label{img:tc-phi}
\end{figure}

\begin{figure}[htbp]
	\centering
	\includegraphics[height=0.45\textheight]{./img/tip-correction/alpha.pdf}
	\caption{Angle of attack distribution with and without tip correction.}
	\label{img:tc-alpha}
\end{figure}

\begin{figure}[htbp]
	\centering
	\includegraphics[height=0.45\textheight]{./img/tip-correction/a.pdf}
	\caption{Axial induction distribution with and without tip correction.}
	\label{img:tc-a}
\end{figure}

\begin{figure}[htbp]
	\centering
	\includegraphics[height=0.45\textheight]{./img/tip-correction/dcp_dmu.pdf}
	\caption{Power coefficient distribution with and without tip correction.}
	\label{img:tc-dcp-dmu}
\end{figure}

\begin{figure}[htbp]
	\centering
	\includegraphics[height=0.45\textheight]{./img/tip-correction/f.pdf}
	\caption{Prandtl's tip loss factor distribution with and without tip correction.}
	\label{img:tc-f}
\end{figure}

\section{Influence of numerical discretization \textcolor{red}{BERNAT}}

\section{Evaluation of stagnation enthalpy \textcolor{green}{CARLOS}}

If heat exchange, viscous forces and compressibility effects are neglected, the flow temperature does not change. Therefore, it can be assumed that the air's internal energy is always constant. Then, the changes in stagnation enthalpy and stagnation pressure are equivalent. The mechanical energy equation (\ref{eq:mech-energy}) describes how the stagnation pressure $ p_t $ changes. Note that it has been obtained as the scalar product of the momentum equation and the velocity vector.

\begin{equation}
	\vec{v} \cdot \vec{\nabla} p_t = \vec{v} \cdot \vec{\nabla} \vec{\vec{\tau}} + \vec{v} \cdot \vec{b}
	\label{eq:mech-energy}
\end{equation}

If viscous forces are not considered, $ \tau = 0 $, the stagnation pressure can only due to the body forces $ b $ work. The only domain region where these forces are not null, and therefore, exert some work is the rotor plane. This means that the stagnation pressure only changes across the rotor plane. See equations (\ref{eq:pt-up}) and (\ref{eq:pt-down}) for the stagnation pressure expressions up and down-stream of the rotor plane respectively. They are plotted in Figure \ref{img:pt}.

\begin{equation}
	p_{t_u} = p_a + \frac{1}{2} \rho u_{\infty}^2
	\label{eq:pt-up}
\end{equation}

\begin{equation}
	p_{t_d} = p_a + \frac{1}{2} \rho u_{\infty}^2 (1-2a)^2
	\label{eq:pt-down}
\end{equation}

\begin{figure}[htbp]
	\centering
	\includegraphics[height=0.45\textheight]{./img/stagnation-enthalpy.pdf}
	\caption{Stagnation enthalpy distribution.}
	\label{img:pt}
\end{figure}

\section{System of circulation and vorticity \textcolor{green}{CARLOS}}

Plot a representation of the system of circulation. Discuss the generation and release of vorticity in relation to the loading and circulation over the blade.

\section{Operational point \textcolor{blue}{NIKLAS}}
